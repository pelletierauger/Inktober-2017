%!TEX root = structure.tex

\section{Introduction}

\section{Les éléments qu'il me faut}
J'ai besoin de trois modes : \textit{drawGraph}, \textit{detectGraph} et \textit{animate}.

Et j'ai besoin d'un ensemble de \textit{systems} entre lesquels je peux alterner à ma guise.

\section{Définition des systèmes}

Les systèmes sont définis par un ensemble de graphes. Les graphes ont des sommets. Ces sommets ont des comportements (\textit{behaviours}). La seule chose que fait l'instance \textit{geo} de p5, c'est d'afficher chacun des sommets de chaque graphes selon sa couleur.

\newpage
\section{Les grenouilles}

L'objet \textit{Grenouille} est à mi-chemin entre une tortue de Papert et une particule classique. Elle est définie par un ensemble d'instructions écrites en \textit{turtle talk}, et cet ensemble est parcouru en boucle. À chaque pas, la nouvelle position de la grenouille est calculée en considérant à la fois les forces extérieures qui l'affectent et cet ensemble d'instructions qui lui est intrinsèque.

L'objet \textit{Grenouille} doit donc avoir une donnée \textit{currentStep} qui est incrémentée à chaque pas puis remis à zéro lorsque toutes les instructions ont été parcourues.

Et à chaque \textit{step}, la position que devrait gagner la grenouille à la fin du pas, cette position à laquelle la grenouille espère accéder, cette position de rêve est considéré comme un attracteur très fort.

Le \textit{dialecte de la grenouille} doit ainsi différer du \textit{turtle talk} : un changement d'angle, que ce soit LEFT ou RIGHT, n'est pas considéré comme un pas. Un pas consiste plutôt à une paire de nombres : l'angle de rotation et la longueur du pas. Chacun de ces deux nombres peut être zéro, mais il doit exister.

Ah, mais en fait, je pourrais parler à ma grenouille en \textit{turtle talk}, et ensuite un programme pourrait convertir le tout en \textit{dialecte de grenouille}.

